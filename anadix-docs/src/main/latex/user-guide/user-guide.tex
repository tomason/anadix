\documentclass{report}

\usepackage[utf8]{inputenc}
\usepackage{hyperref}

\title{Anadix user guide}
\author{Tomas Schlosser}


\begin{document}
\maketitle
\tableofcontents

\chapter{Overview}
Anadix is a tool for analyzing HTML files for Section 508 compliance. It uses
Drools rule engine as a core that is used to create classes from parsed HTML
elements and to evaluate rules checking Section 508 compliance. Anadix can be
extended to analyze any source using any set of conditions as long as source and
conditions share the same domain. You can for example change the parser that is
used for HTML file parsing with custom written parser using different HTML
parser implementation. Anadix also provides integration with different tools -
currently it is only Apache Ant, but integration with Selenium is planned.

\chapter{Getting started}
To start using anadix you need only Java SE 6 or above. But if you want to build
Anadix from source you will need Git and Apache Maven as well. 
\section{Requiered environment}
Only a few prerequisities are required for 
\begin{description}
  \item[Java SE 6] is a runtime environment and Anadix can't run without it 
  (\url{http://www.java.com/getjava/}).
  \item[Git] is source control manger that is needed only if you want to build
  Anadix from sources (\url{http://git-scm.com/}).
  \item[Apache Maven] is a project management tool that is needed only if you
  want to build Anadix from sources (\url{http://maven.apache.org/}).
\end{description}

\section{Acquiring binaries}
Basically there are two ways to obtain binary archives (jar) to run Anadix
analyzer.
\subsection{Download binaries from Github}
The fastest way to get Anadix binaries is to download zip file from Github
(\url{http://github.com/tomason/anadix/downloads}, 
\url{http://github.com/downloads/tomason/anadix/anadix-distribution-0.4.0.zip}).

\subsection{Build Anadix from sources}
If you want to get the latest version or you want to build the binaries
yourself or you want to change something you should download the sources and
build the binaries from sources.
\begin{itemize}
  \item Clone the git repository:
    \begin{verbatim}
git clone git@github.com:tomason/anadix.git
	\end{verbatim}
  \item Change working directory:
	\begin{verbatim}
cd anadix
	\end{verbatim}
  \item Checkout tag or branch (optional):
    \begin{verbatim}
git checkout anadix-0.4.0
    \end{verbatim}
  \item Build binaries using Apache Maven:
    \begin{verbatim}
mvn clean install
	\end{verbatim}
\end{itemize}
Binaries obtained this way can be found in directory
'anadix-distribution/target' in ananadix-distribution-0.4.0.zip archive.

\section{Usage}
Anadix is an engine for analyzing and has no user interface so you have to embed
the libraries in your application. There are two ways to get the binaries to
embed the libreries to your application.
\begin{description}
  \item[Copy to classpath] required libraries and then use them in your
  application.
  \item[Use Apache Maven] to manage dependencies of your application. However
  please note that Anadix is not in central Maven repositories and therefore you
  have to build it locally using:
    \begin{verbatim}
mvn clean install
	\end{verbatim}
  If someone else wants to use your project then they must build the anadix from
  sources as well.
  Adding Anadix to your Maven project is easy:
    \begin{verbatim}
<dependency>
  <groupId>org.anadix</groupId>
  <artifactId>anadix-section508</artifactId>
  <version>0.4.0</version>
</dependency>
    \end{verbatim}
  Since section508 has all the necessary libraries as transitive dependencies
  (api, core, html, swingparser) you don't need to specify any other
  dependencies.
\end{description}

\chapter*{Usage}
blah blah blah
\section*{API}
org.anadix.*
\section*{Analysis}

\section*{Reports}

\section*{Custom classes}

\chapter*{Integration}
\section*{Ant}
\section*{Selenium}


\end{document}
