\documentclass{report}

\usepackage[utf8]{inputenc}
\usepackage{hyperref}

\title{Anadix user guide}
\author{Tomas Schlosser}


\begin{document}
\maketitle
\tableofcontents

\chapter{Overview}
Anadix is a tool for analyzing HTML files for Section 508 compliance. It uses
Drools rule engine as a core that is used to create classes from parsed HTML
elements and to evaluate rules checking Section 508 compliance. Anadix can be
extended to analyze any source using any set of conditions as long as source and
conditions share the same domain. You can for example change the parser that is
used for HTML file parsing with custom written parser using different HTML
parser implementation. Anadix also provides integration with different tools -
currently it is only Apache Ant, but integration with Selenium is planned.

\chapter{Getting started}
To start using Anadix you need only Java SE 6 or above. But if you want to build
Anadix from source you will need Git and Apache Maven as well. 
\section{Required environment}
Only a few prerequisites are required for successfully running Anadix:
\begin{description}
  \item[Java SE 6] is a runtime environment and Anadix can't run without it 
  (\url{http://www.java.com/getjava/}).
  \item[Git] is source control manger that is needed only if you want to build
  Anadix from sources (\url{http://git-scm.com/}).
  \item[Apache Maven] is a project management tool that is needed only if you
  want to build Anadix from sources (\url{http://maven.apache.org/}).
\end{description}

\section{Acquiring binaries}
Basically there are two ways to obtain binary archives (jar) to run Anadix
analyzer.
\subsection{Fork a repository from Github}
The best option in case you want to change the code and then contribute it back to the project. For information on how to do that see \url{http://help.github.com/fork-a-repo/}.

\subsection{Download binaries from Github}
The fastest way to get Anadix binaries is to download zip file from Github
(\url{http://github.com/tomason/anadix/downloads}, 
\url{http://github.com/downloads/tomason/anadix/anadix-distribution-1.0.0.zip}).

\subsection{Build Anadix from sources}
If you want to get the latest version or you want to build the binaries
yourself or you want to change something you should download the sources and
build the binaries from sources.
\begin{itemize}
  \item Clone the git repository:
    \begin{verbatim}
git clone git@github.com:tomason/anadix.git
	\end{verbatim}
  \item Change working directory:
	\begin{verbatim}
cd anadix
	\end{verbatim}
  \item Checkout tag or branch (optional):
    \begin{verbatim}
git checkout anadix-1.0.0
    \end{verbatim}
  \item Build binaries using Apache Maven:
    \begin{verbatim}
mvn clean install
	\end{verbatim}
\end{itemize}
Binaries obtained this way can be found in directory
'anadix-distribution/target' in anadix-distribution-1.0.0.zip archive.

\section{Usage}
Anadix is an engine for analyzing and has no user interface so you have to embed
the libraries in your application. There are two ways to get the binaries to
embed the libraries to your application.
\begin{description}
  \item[Copy to classpath] required libraries and then use them in your
  application.
  \item[Use Apache Maven] to manage dependencies of your application. Adding Anadix to your Maven project is easy:
    \begin{verbatim}
<dependency>
  <groupId>org.anadix</groupId>
  <artifactId>anadix-section508</artifactId>
  <version>1.0.0</version>
</dependency>
    \end{verbatim}
  Since section508 has all the necessary libraries as transitive dependencies
  (api, core, html, swingparser) you don't need to specify any other
  dependencies.
\end{description}

\chapter{Functionality}
Currently Anadix offers an engine to analyze whatever can be analyzed. For that
purpose Drools rule engine is used. In the basic configuration only a Section
508 compliance of web pages is provided but it is possible to write parser for
any kind of data you can think of (e.g. read data from files, database,
file system) and condition set for any purpose. Section 508 compliance testing
bring the parser of HTML files and HTML domain that can be reused, replaced or
just extended.

\section{API}
API consists of several basic interfaces and two classes:
\begin{description}
\item[org.anadix.Anadix] creates instances of Analyzer and to set default
implementations of most important interfaces.
\item[org.anadix.factories.SourceFactory] creates Sources for analysis from
different types of sources (URL, Stream, Reader, String). Every creating sources (except the one for StringSource) has two methods~-- one with the source parameter only and the other with source parameter and boolean value. The sources are cached by default (the whole source is read and text is saved into StringSource that is stored in memory), caching the sources can be turned off by passing false value as a second parameter to the source creation method. IMPORTANT: Use caching only for text resources!
\item[org.anadix.Analyzer] provides only one simple method - analyze that takes
parameter Source and produces Report.
\item[org.anadix.ElementFactory] serves several purposes: ensures the Parser and
ConditionSet use the same domain, creates instances of elements, keeps the
parsed elements. ElementFactory used is determined from Parser and ConditionSet
classes that both have getElementFactoryClass() method.
\item[org.anadix.Parser] parses the Source given to it through Analyzer
instance. This class is meant to take the Source (document, file, database),
break it to elements (lines, tags, attributes, rows) and inserts those elements
to ElementFactory.
\item[org.anadix.ConditionSet] takes elements from the original source and
analyzes their relationship and reports predefined patterns.
\item[org.anadix.Report] contains reports generated by conditions and divides
them to categories by item status (error, warning, ok, info, manual).
\item[org.anadix.ReportFormatter] formats the report generated by analysis to
human readable form.
\end{description}

\section{Analysis}
Analysis is performed by calling methods defined in classes implementing API
interfaces:
\begin{itemize}
  \item Source is passed to Analyzer.analyze(Source) by user.
  \item Source is further passed to Parser.parse(Source) that inserts the parsed
  elements to ElementFactory.
  \item Rules defined in ConditionSet are evaluated.
  \item Report is created by picking up all ReportItem instances.
  \item Report is returned to user.
\end{itemize}
Currently all the rules are fired at once (both Parser rules and ConditionSet
ones) but in future some mechanism will be enforced to separate parsing from
evaluating conditions.

\section{Reports}
Report class returned by analysis holds all ReportItems in separate collections
per ItemStatus. ReportItem contains several fields:
\begin{description}
  \item[Status] item status (mandatory).
  \item[Item text] short description of report item (mandatory).
  \item[Description] longer description of report item.
  \item[Advice] field for advice what to do with the report or (in case of
  error or warning) how to fix it.
\end{description}
Reports hold all the information necessary to create a human readable report.
That is done through ReportFormatter class that can generate any form of output.
Currently only a very simple formatter is implemented that is best used for
debugging purposes but holds no real value. This will be addressed in future
versions.

\section{Extensibility}
Since Anadix is a very simple engine and does very little it can be extended to
any purpose. Any implementation can be replaced or extended.

With distribution of Anadix come three modules:
\begin{description}
  \item[anadix-html] provides HTML domain. The classes are simple POJOs very
  easily usable from Drools rules engine. The classes and attributes have been
  taken from HTML 4.01 Transitional DTD so the domain can be reused in any
  analysis of HTML 4.01 Transitional documents.
  \item[anadix-swingparser] provides HTML parser. It is based on parser from
  package javax.swing.text.html.parser. It can be used for parsing all HTML
  documents.
  \item[anadix-section508] provides conditions of web page compliance with
  Section 508.
\end{description}

To use Anadix for different purpose you just have to implement or reuse domain,
parser and condition set that suites your needs. There are only a few things you
should keep in mind:
\begin{description}
  \item[Use the same domain] for parser and condition set. The ElementFactory
  used by ConditionSet must be assignable to the one used by Parser. This is
  checked during creation of analyzer and will throw an exception.
  \item[Take advantage of Drools] as it offers a lot of functionality that can
  come in handy when analyzing relationship between elements (backward
  chaining, traits). It can also make parsing easier if you use templates or
  decision tables.
\end{description}

\chapter{Integration}
To make using Anadix easier it was integrated with some tools.
\section{Ant}
The ant task definition is available in package anadix-ant.jar and provides a
way to analyze a batch of files resulting in reports named accordingly to the
source file.

\section*{Selenium}
Integration with Selenium is planned as the first priority is to analyze web
pages. It will store the web pages after every click and analyze them by users
decision - right after storing, when the method is called or never letting user
analyze the sources himself.

\end{document}
